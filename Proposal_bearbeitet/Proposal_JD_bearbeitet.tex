\documentclass[10pt]{article}

\usepackage[utf8]{inputenc}
\usepackage[english]{babel}
\usepackage{multicol}
\usepackage{enumitem}


\begin{document}
\begin{titlepage}
\begin{center}

%----------------------------
%Titlepage
%----------------------------

A proposal for the Machine Learning Application\\
Research Paper\\
\vspace{4mm}
\today
\vspace{4mm}
\end{center}
\begin{center}
\textbf{\Large Optimising online documents for fact-checking}\\
\vspace{15mm}
\textbf{Group:}\\
\vspace{2mm}
Alpha \\
\vspace{15mm}
\textbf {Members:}\\
\vspace{2mm}
\end{center}
\begin{tabular}{ l c r }
  Alexander Peikert & alexpeikert@uni-koblenz.de & 218200812 \\
  Clemens Steinmann & csteinmann@uni-koblenz.de & 218200209 \\
  Erwin Letkemann & erwinletkemann@uni-koblenz.de & 218200352 \\
  Julian Dillmann & juliandillmann@uni-koblenz.de & 218100919 \\
\end{tabular}
\vfill
\begin{center}
\vspace{8mm}
WeST \\
Universität Koblenz-Landau \\
Germany \\
\vspace{2mm}
\end{center}
\clearpage
\end{titlepage}

\begin{multicols}{2}

%-------------------------
%Intro
%-------------------------

\section{Introduction}
The aim of the application is to extract content of published news articles and to assign them various \grqq{}Information nutritional labels\grqq{}, based on \ref{[1]}.\\
The web application can be used by different kind of users:
\begin{enumerate}
\item Ordinary web users can find similar articles for a given news article. The received labels can help answering the question whether the article was trustworthy or not. If latter, the user can continue reading on the topic with articles from different sources suggested by the application.
\item Other researchers and fact-checkers can use our application, especially the labels, as a basis on further research on automation of fact-checking.
\end{enumerate}
The labels we are focusing on will be at least 3 of the following: factuality / opinion, readability, emotion, controversy, credibility.

%--------------------------
%Application
%--------------------------

\section{Application}
\subsection{Components}
\begin{itemize}
\item Database
\item Similarity checker / finder
\item Content extractor
\item Content NLP (Natural language processing)
\item Web-application
\end{itemize}
\subsection{General procedure}
Given an input URL to an English news article, the applications' database will check whether the url has already been checked or not.
\begin{enumerate}
\item If the news article is present in our database we will output the already calculated labels and the x-best matching articles in our database.
\item Otherwise the news article content extractor will crawl the html to extract the plaintext of the news article and each of the labels will be evaluated with corresponding natural language processing methods. The result will be added to the database which leads us to 1.
\end{enumerate}

\subsection{Qualities of the components}
\begin{itemize}
\item The \textbf{database} should hold: URL, metadata, labels, topic, published date, and should be quickly accessible based on topics to judge relevance of articles given another article.
\item \textbf{Similarity} between two news articles describes the general topic in our use case. Consequently best matching news articles are those which have the highest lexical similarity and semantic similarity.
\item \textbf{Labels} don't need to be specifically trained of/for news articles. Those are general purpose algorithms for each individual task.
\item Possibly include the most used emotions to indicate discussion on opposite viewpoints within an articles. (See basic idea of Emotion)
\end{itemize}

\subsection{Basic ideas}
subject to change
\begin{itemize}
\item Factuality/ Opinion: \ref{[1]} NLP to calculate ratio of factual to opinion sentences.
\item Readability:  \ref{[2]} use some existing algorithm on the text.
\item Emotion: \ref{[3]}, \ref{[4]} NLP of sentences and sum will tell us the ratio of positive and negative emotions given in the text.
\item Controversy: \ref{[5]} checking for the number included Topics over the text.
\item Credibility: \ref{[6]} (or possibly some other existing website) compare to metadata.
\end{itemize}


%----------------------------
%Related Work
%----------------------------

\section{Related Work}
\begin{enumerate}[label={[\arabic*]}]
\item Fuhr, Norbert et al. \grqq{}\textbf{An Information Nutritional Label for Online Documents.}\grqq{} SIGIR Forum 51 (2018): 46-66.\label{[1]}
\item Sahu, Ishan \& Majumdar, Debapriyo. (2017). \textbf{Detecting Factual and Non-Factual Content in News Articles.} 1-12. 10.1145/3041823.3041837.\label{[2]}
\item \textbf{AFFIN Database} Informatics and Mathematical Modelling, Technical University of Denmark (2011) \label{[3]}
\item Cambria, Erik et al. “\textbf{The Hourglass of Emotions.}” COST 2102 Training School (2011). \label{[4]}
\item "Wikipedia:List of controversial issues" Wikipedia: The Free Encyclopedia. Wikimedia Foundation, Inc. 18 March 2020, Web. 22 March 2020, en.wikipedia.org/wiki/Wikipedia: List\_of\_controversial\_issues \label{[5]}
\item "Wikipedia:Reliable sources/Perennial sources" Wikipedia: The Free Encyclopedia. Wikimedia Foundation, Inc. 21 March 2020, Web. 22 March 2020, en.wikipedia.org/wiki/Wikipedia: Reliable\_sources/Perennial\_sources \label{[6]}
\end{enumerate}

  


\end{multicols}
\end{document}
